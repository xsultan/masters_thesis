% -*- root: cuthesis_masters.tex -*-

\section{Summary of Addressed Topics}


Chapter 2 offers a synopsis of the latest research on technical debt, which has generated much interest in the software development research community in recent years. Consequently, we find it to be an opportune time to pick up where this body of research left off and address some of the inquiries it has not yet delved into, whether not thoroughly enough or not at all. It should do much in the way of informing current debate in the field to determine whether the technical debt metaphor and developers' views of the practice hold up under further scrutiny.

Chapter 3 presents the impact of comment-based technical debt (\SATD) on software quality. In this chapter, we analyze the source code comments of five well-commented open-source projects representing various domains and programming languages that have a large number of contributors. We find that (i) files with \SATD have more defects than files without \SATD, (ii) SATD changes are associated with less future defects than non-SATD changes and (iii) SATD changes are more difficult to perform.

Chapter 4 presents the effects of comment- versus metric-based technical debt on software quality. In this chapter, we analyze forty open-source projects to understand how god classes and \SATD influence software quality. We observe that (i) neither the incidence of \SATD nor god files was correlated with defects, (ii) future defects are introduced at a higher rate by god and SATD changes and (iii) the difficulty imposed on the system is greater for god and SATD changes.

\section{Contributions}
The major contributions of this thesis are as follows:

\begin{itemize}
	\item A diachronic survey of the state of the art in technical debt detection: We supply a comprehensive account of the technical debt metaphor's inception and popularization. Specifically, we study the impact of metric- and comment-based approaches on software quality.
\end{itemize}
\section{Future Work}

