% -*- root: cuthesis_masters.tex -*-

\section{Summary of Addressed Topics}


Chapter 2 offers a synopsis of the latest research on technical debt, which has generated much interest in the software development research community in recent years. Consequently, we find it to be an opportune time to pick up where this body of research left off and address some of the inquiries it has not yet delved into, whether not thoroughly enough or not at all. It should do much in the way of informing current debate in the field to determine whether the technical debt metaphor and developers' views of the practice hold up under further scrutiny.

Chapter 3 presents the impact of comment-based technical debt (\SATD) on software quality. In this chapter, we analyze the source code comments of five well-commented open-source projects representing various domains and programming languages that have a large number of contributors. We find that (i) files with \SATD have more defects than files without \SATD, (ii) SATD changes are associated with less future defects than non-SATD changes and (iii) SATD changes are more difficult to perform.

Chapter 4 presents the effects of comment- versus metric-based technical debt on software quality. In this chapter, we analyze forty open-source projects to understand how god classes and \SATD influence software quality. We observe that (i) neither the incidence of \SATD nor god files was correlated with defects, (ii) future defects are introduced at a higher rate by god and SATD changes and (iii) the difficulty imposed on the system is greater for god and SATD changes.

\section{Contributions}
The major contributions of this thesis are as follows:

\begin{itemize}
	\item A diachronic survey of the state of the art in technical debt detection: We supply a comprehensive account of the technical debt metaphor's inception and popularization. Specifically, we study the impact of metric- and comment-based approaches on software quality.
\end{itemize}
\section{Future Work}

We believe that our thesis advances the state of the art in measuring the impact of technical debt on software quality. Though our research clarifies the dynamics of this complex relationship, there are other dimensions of software quality that should be navigated in order to understand the full force of technical debt's impact.

\subsection{Automating Technical Debt Management}

Our findings lay the groundwork for creating a tool that would assist developers in understanding and mitigating the undesirable long-term consequences of incurring technical debt. We have every reason to believe that a tool of this kind would facilitate detection and management of different varieties of technical debt while enhancing design practices, which would optimize the overall quality of the system and dovetail formerly discrete stages in the development process.


\subsection{Diversifying Code Smell Representation}

The scope of this thesis was not conducive to studying the overlap between \SATD and all instantiations of code smells, yet developers would certainly benefit from further research that demonstrates how code smells besides god classes fit into the picture. The overlap between \SATD and lazy class, black sheep, shotgun surgery, etc. could indicate that comment-based approaches to detecting technical debt are more or less reliable than the respective metric-based approaches.

\subsection{Granularizing Technical Debt Classification}

We have focused our attention so far on the impact of technical debt on software quality at the file and change levels. Naturally, a logical progression would be to accommodate the method level, as it would provide more granular insights into the implications of technical debt for quality as a result of increasing confidence in the organization of files into SATD and non-SATD categories.
