% -*- root: cuthesis_masters.tex -*-  

% What TD is

Software companies and organizations have a common goal when developing software projects---to deliver high-quality, useful software in a timely manner. However, in most practical settings developers and development companies are saddled with deadlines, urging them to release earlier than the ideal date in terms of product quality. Such situations are all too common and in many cases force developers to take shortcuts \cite{kruchten2013technical} \cite{seaman2015technical}. Recently, the term \emph{technical debt} was coined to represent the phenomenon of ``doing something that is beneficial in the short term but will incur a cost later on''~\cite{cunningham1993wycash}. Prior work has shown that there are many different reasons why practitioners assume technical debt. These include: a rush to deliver a software product given a tight schedule, deadlines to incorporate with a partner product before release, time-to-market pressure and incentives to satisfy customer demands in a time-sensitive industry~\cite{lim2012balancing} that still expects them to meet its software quality standards.
\par

Most definitions of software quality recognize two subdivisions: external quality and internal quality. In one such definition, Fitzpatrick~\cite{fitzpatrick1996software} characterizes software quality as ``the extent to which an industry-defined set of desirable features are incorporated into a product so as to enhance its lifetime performance." How the features that end users enjoy conform to their individual preferences determines external quality. Internal quality, by contrast, is a composite evaluation of the features developers have built into the code on the production side. While industries differ on how they weight the specific features in the development process (quality factors), there is consensus that maintainability goes a long way in determining internal quality.

Studies over the years have proposed different approaches to measure technical debt, which has been found to impact (internal) quality. Zazworka \textit{et al.} \cite{zazworka2011investigating}, for instance, recommend combining automated technical debt detection tools with manual detection strategies. For his part, Marinescu \cite{marinescu2004detection} has proposed a technique to detect code smells, specifically god classes, based on sets of thresholds defined on various object-oriented metrics.\par

God classes typically exhibit high complexity and low inner-class cohesion and access foreign class data at a higher rate than the trivial classes whose workloads they consolidate. Moreover, god classes withhold tasks that would otherwise be delegated elsewhere \cite{lanza2007object}. Their size and many dependencies make the system harder to comprehend and more defect-prone \cite{fowler1999refactoring}.

% People studied different aspects of TD
More recently, a study by Potdar and Shihab \cite{ICSM_PotdarS14} introduced a new way to identify \SATD (SATD) through source code comments. SATD is technical debt that developers themselves report through source code comments. Prior work \cite{MTD15p9} has demonstrated that accrual of SATD is commonplace in software projects, where implementing comment-based approaches can identify different types of technical debt (e.g. design, defect, and requirement debt), just as metric-based approaches detect technical debt using static analysis tools. Today, these two approaches are the state of the art in measuring technical debt.\par



Intuition and general belief concur that technical debt negatively impacts software maintenance and overall quality~\cite{zazworka2011investigating,spinola2013investigating,GuoSGCTSSS11,seaman2015technical,kruchten2013technical}. However, to the best of our knowledge, there is little empirical evidence as to how SATD and metric-based technical debt are related to software quality. Such a study is critical since it will help us either confirm or refute entrenched preconceptions regarding the technique and better understand how to manage SATD and metric-based technical debt.\par



% In this paper we do 1, 2,3
Since there is no prior work on the relationship between SATD and quality, we conduct a preliminary study in Chapter 3 of this thesis to empirically investigate the relationship between SATD and software quality in five open-source projects. In particular, we examine whether (i) files with SATD have more defects compared to files without SATD, (ii) whether SATD changes introduce more future defects and (iii) whether SATD-related changes tend to be more difficult. We measure the difficulty of a change in terms of the amount of churn, the number of files it touches, the number of modified modules and its entropy. \par

Having studied the relationship between SATD and software quality, we then expand our study to include another measure of technical debt (metric-based technical debt). Such a study is important for providing researchers and practitioners with different observations of technical debt; comparing the new approach to the traditional approach in terms of how they relate to software quality and advancing the state of the art in understanding and mitigating technical debt.\par



Therefore, in Chapter 4 of this thesis, we compare the SATD and metric-based approaches across 40 open-source systems in order to validate our preliminary findings with a larger data set. Specifically, we compare: (i) the defects of god and SATD files versus non-god and non-SATD files, (ii) the future defect introduction of god and SATD changes versus non-god and non-SATD changes and (iii) the difficulty of god and SATD changes versus non-god and non-SATD changes. In addition, we measure (iv) the overlap between metric- and comment-based technical debt files.

 
\section{Thesis Overview}

\textbf{Chapter 2: Literature Review:} This chapter synthesizes more detailed discussions of the technical debt metaphor from websites, blogs and research papers to provide a brief chronological survey of the most prevalent of its various applications, including some of the most recent.  At the end of this chapter, we offer a critical assessment of the current status of technical debt in the field, including its reputation among software developers and the drawbacks it has been found to entail.

\textbf{Chapter 3: Examining the Relationship Between Self-Admitted Technical Debt and Software Quality:} We preliminarily examine how \SATD relates to software quality across five open-source projects (Chromium, Cassandra, Spark, Tomcat and Hadoop) on three accounts: (i) which of SATD and non-SATD files have more existing defects, (ii) which of SATD and non-SATD changes induce more future defects and (iii) which of SATD and non-SATD changes are more difficult to execute. We adhere to precedent in measuring change difficulty using amount of churn, number of files, number of modified modules and change entropy. Our findings demonstrate: (i) no clear trend relating \SATD and existing defects, (ii) a higher incidence of future defects for non-SATD changes and (iii) greater difficulty in performing SATD changes. Therefore, based on our findings, we conclude that \SATD adversely affects system maintenance by increasing change complexity but is dissociated from defects.

\textbf{Chapter 4: Comparing the Relationship Between Comment- Versus Metric-Based Technical Debt and Software Quality:} We conduct a wide-ranging study on 40 open-source projects to investigate the ways in which code smells (God Classes) and \SATD influence software quality and concentrate on three points of view: (i) whether god and SATD files have more defects than non-god and non-SATD files, (ii) to what extent god and SATD changes are correlated with future defects and (iii) whether performing god and SATD changes imposes more difficulty on the system, where difficulty is measured by amount of churn, number of affected files and modified modules and change entropy. In the interest of comparing the approaches, we also determine: (iv) to what extent the metric- and comment-based approaches identify the same sources of technical debt. We conclude that: (i) neither god nor SATD files tend to have more defects than non-god and non-SATD files, (ii) god- and SATD-related changes tend to induce a greater number of future defects and (iii) god- and SATD-related changes are more difficult to perform than non-god and non-SATD changes. Thus, god classes and \SATD are detrimental insofar as they increase future defects and change complexity. We also found that (iv) the metric- and comment-based approaches complement each other at a rate of 11\% to 34\%.

\section{Thesis Contributions}
The major contributions of this thesis are as follows:
\begin{itemize}
	\item Empirically examine the relationship between \SATD and software quality.
	\item Enhance knowledge of the technical debt phenomenon by presenting a large-scale empirical study that compares the SATD (comment-based) and non-SATD (metric-based) approaches.
	\item Provide evidence that technical debt tends to induce more future defects and increase system complexity.
\end{itemize}