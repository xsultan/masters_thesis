% -*- root: cuthesis_masters.tex -*-  

% What TD is
\todo{This part has to be re-written after the completion of chapter 4}
Software companies and organizations have a common goal when developing software projects - to deliver high-quality, useful software in a timely manner. However, in most practical settings developers and development companies are saddled with deadlines, urging them to release earlier than the ideal date in terms of product quality. Such situations are all too common and in many cases force developers to take shortcuts \cite{kruchten2013technical} \cite{seaman2015technical}. Recently, the term \emph{technical debt} was coined to represent the phenomenon of ``doing something that is beneficial in the short term but will incur a cost later on''~ \cite{cunningham1993wycash}. Prior work has shown that there are many different reasons why practitioners assume technical debt. These reasons include: a rush to deliver a software product given a tight schedule, deadlines to incorporate with a partner product before release, time-to-market pressure, as well as meeting customer demands in a time-sensitive industry~\cite{lim2012balancing}.


% People studied different aspects of TD
More recently, a study by Potdar and Shihab \cite{ICSM_PotdarS14} introduced a new way to identify technical debt through source code comments, referred to as self-admitted technical debt (SATD). SATD is technical debt that developers themselves report through source code comments. Prior work \cite{MTD15p9} has demonstrated that accrual of SATD is commonplace in software projects, where its implementation can identify different types of technical debt (e.g., design, defect, and requirement debt).

Intuition and general belief concur that such rushed development tasks (also known as technical debt) negatively impact software maintenance and overall quality~\cite{zazworka2011investigating,spinola2013investigating,GuoSGCTSSS11,seaman2015technical,kruchten2013technical}. However, to the best of our knowledge, there is no empirical study that examines the impact of SATD on software quality. Such a study is critical since it will help us (i) confirm or refute entrenched preconceptions regarding the technique and (ii) better understand how to manage SATD.

% In this paper we do 1, 2,3
Therefore, in this thesis, we empirically investigate the relation between SATD and software quality in forty-five open-source projects. In particular, we examine whether (i) files with SATD have more defects compared to files without SATD, (ii) whether SATD changes introduce future defects, and (iii) whether SATD-related changes tend to be more difficult. We measured the difficulty of a change in terms of the amount of churn, the number of files, the number of modified modules in a change, as well as, entropy of a change. We perform our study on forty-five open-source projects. Our findings show that: i) there is no clear relationship between defects and SATD. While it is true that SATD files have more bug-fixing changes in a number of the studied projects, in other projects, files without SATD have more defects, ii) SATD changes are associated with less future defects than non-technical debt changes, yet iii) SATD changes (i.e., changes touching SATD files) are more difficult to perform. Our study indicates that although technical debt may have negative effects, its impact is not related to defects, rather its impact is in making the system more difficult to change in the future.



\sultan{Add motivation and reference for non SATD}



\section{Research Hypothesis}
Prior research lead us to the formation of our research hypothesis. We believe that:

\conclusionbox{The impact of \SATD on software quality \sultan{?}}  

\section{Thesis Overview}

\textbf{Chapter2: Literature Review:} The technical debt metaphor has come to encompass miscellaneous "quick fixes" and non-optimal solutions since its inception.  This chapter synthesizes more detailed discussions of the technical debt metaphor from websites, blogs, and research papers to provide a brief chronological survey of the most prevalent of its various applications, including some of the most recent.  At the end of this chapter, we offer a critical assessment of the current status of technical debt in the field, as well as its reputation among software developers and the drawbacks it potentially entails.

\textbf{Chapter3: Examining the Impact of Self-admitted Technical Debt on Software Quality:} We empirically examine how \SATD impacts software quality across five open-source projects (Chromium, Cassandra, Spark, Tomcat and Hadoop) on three accounts: (i) which of SATD and non-SATD files have more existing defects, (ii) which of SATD and non-SATD changes induce more future defects and (iii) which of SATD and non-SATD changes are more difficult to execute. We adhere to precedent in measuring change difficulty using amount of churn, number of files, number of modified modules and change entropy. Our findings demonstrate (i) no clear trend relating \SATD and existing defects, (ii) a higher incidence of future defects for non-SATD changes and (iii) greater difficulty in performing SATD changes. Therefore, \SATD adversely affects system maintenance by increasing change complexity but is dissociated from defects.

\textbf{Chapter4: Comparing the Impact of Comment- vs. Metric-Based Technical Debt:} We pool forty open-source projects to investigate the ways in which \SATD and code smells (god classes) influence software quality and concentrate on three points of view: (i) whether god and SATD files have more defects than non-god and non-SATD files, (ii) to what extent god and SATD changes are correlated with future defects and (iii) whether performing god and SATD changes imposes more difficulty on the system, where difficulty is measured by amount of churn, number of affected files and modified modules and entropy of the change. We conclude that (i) god and SATD files are uncorrelated with defects, (ii) god and SATD changes induce a greater number of future defects and (iii) god and SATD changes make the system more complex. Thus, god classes and \SATD are detrimental insofar as they increase future defects and change complexity.

\section{Thesis Contributions}
The major contributions of this thesis are as follows:
\begin{itemize}
	\item 
	\item
\end{itemize}