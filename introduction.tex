% -*- root: cuthesis_masters.tex -*-  

% What TD is
Software companies and organizations have a common goal when developing software projects---to deliver high-quality, useful software in a timely manner. However, in most practical settings developers and development companies are saddled with deadlines, urging them to release earlier than the ideal date in terms of product quality. Such situations are all too common and in many cases force developers to take shortcuts \cite{kruchten2013technical} \cite{seaman2015technical}. Recently, the term \emph{technical debt} was coined to represent the phenomenon of ``doing something that is beneficial in the short term but will incur a cost later on''~\cite{cunningham1993wycash}. Prior work has shown that there are many different reasons why practitioners assume technical debt. These reasons include: a rush to deliver a software product given a tight schedule, deadlines to incorporate with a partner product before release, time-to-market pressure, as well as incentives to satisfy customer demands in a time-sensitive industry~\cite{lim2012balancing}.

Studies over the years have proposed different approaches to measure technical debt. Zazworka \textit{et al.} \cite{zazworka2011investigating}, for instance, recommend combining automated technical debt detection tools with manual detection strategies. For his part, Marinescu \cite{marinescu2004detection} has proposed a technique to detect code smells, specifically God Classes, based on sets of thresholds defined on various object-oriented metrics.

% People studied different aspects of TD
More recently, a study by Potdar and Shihab \cite{ICSM_PotdarS14} introduced a new way to identify technical debt through source code comments, referred to as self-admitted technical debt (SATD). SATD is technical debt that developers themselves report through source code comments. Prior work \cite{MTD15p9} has demonstrated that accrual of SATD is commonplace in software projects, where its implementation can identify different types of technical debt (e.g., design, defect, and requirement debt).\par

Intuition and general belief concur that technical debt negatively impacts software maintenance and overall quality~\cite{zazworka2011investigating,spinola2013investigating,GuoSGCTSSS11,seaman2015technical,kruchten2013technical}. However, to the best of our knowledge, there is no empirical study that examines the impact of SATD on software quality. Such a study is critical since it will help us either confirm or refute entrenched preconceptions regarding the technique and better understand how to manage SATD.\par

% In this paper we do 1, 2,3
Therefore, in chapter 3 of this thesis, we empirically investigate the relationship between SATD and software quality in five open-source projects. In particular, we examine whether (i) files with SATD have more defects compared to files without SATD, (ii) whether SATD changes introduce more future defects and (iii) whether SATD-related changes tend to be more difficult. We measured the difficulty of a change in terms of the amount of churn, the number of files it touches, the number of modified modules and its entropy. \par

Having studied the impact of comment-based technical debt on software quality, we now expand our study to include another measure of technical debt (metric-based technical debt). Such a study is important for providing researchers and practitioners with different observations of technical debt; comparing the new approach to the traditional approach on their impact on software quality and advancing the state of the art in understanding and mitigating technical debt.\par


Therefore, in chapter 4 of this thesis, we apply the comment- and metric-based approaches to 40 open-source systems in order to compare (i) the defects of god and SATD files versus non-god and non-SATD files, (ii) the future defect introduction of god and SATD changes versus non-god and non-SATD changes and (iii) the difficulty of god and SATD changes versus non-god and non-SATD changes. In addition, we measure (iv) the overlap between metric- and comment-based technical debt files.




\section{Research Hypothesis}
Prior research has led us to the formation of our research hypothesis. We believe that:

\conclusionbox{Though the impact of metric-based technical debt, identified by static analysis tools, has been studied extensively, little research has been done to determine the impact of comment-based technical debt (\SATD) on software quality.  We hypothesize that comment-based technical debt impacts software quality negatively and complements metric-based technical debt. Since metric-based identification of technical debt requires expensive and in-depth analysis, we believe that it would be worthwhile to confirm sources of technical debt via a comment-based approach in order to minimize metric-based false positives.}
 
\section{Thesis Overview}

\textbf{Chapter 2: Literature Review:} This chapter synthesizes more detailed discussions of the technical debt metaphor from websites, blogs and research papers to provide a brief chronological survey of the most prevalent of its various applications, including some of the most recent.  At the end of this chapter, we offer a critical assessment of the current status of technical debt in the field, as well as its reputation among software developers and the drawbacks it potentially entails.

\textbf{Chapter 3: Examining the Impact of Self-Admitted Technical Debt on Software Quality:} We empirically examine how \SATD impacts software quality across five open-source projects (Chromium, Cassandra, Spark, Tomcat and Hadoop) on three accounts: (i) which of SATD and non-SATD files have more existing defects, (ii) which of SATD and non-SATD changes induce more future defects and (iii) which of SATD and non-SATD changes are more difficult to execute. We adhere to precedent in measuring change difficulty using amount of churn, number of files, number of modified modules and change entropy. Our findings demonstrate (i) no clear trend relating \SATD and existing defects, (ii) a higher incidence of future defects for non-SATD changes and (iii) greater difficulty in performing SATD changes. Therefore, based on our findings, we conclude that \SATD adversely affects system maintenance by increasing change complexity but is dissociated from defects.

\textbf{Chapter 4: Comparing the Impact of Comment- vs. Metric-Based Technical Debt:} We study 40 open-source projects to investigate the ways in which code smells (god classes) and \SATD influence software quality and concentrate on three points of view: (i) whether god and SATD files have more defects than non-god and non-SATD files, (ii) to what extent god and SATD changes are correlated with future defects and (iii) whether performing god and SATD changes imposes more difficulty on the system, where difficulty is measured by amount of churn, number of affected files and modified modules and change entropy. In the interest of comparing the approaches, we also determine (iv) to what extent the metric- and comment-based approaches identify the same sources of technical debt. We conclude that: (i) both god and SATD files are uncorrelated with defects, (ii) god-related and SATD-related changes induce a greater number of future defects and (iii) god and SATD-related changes make the system more complex. Thus, god classes and \SATD are detrimental insofar as they increase future defects and change complexity. We also found that (iv) the metric- and comment-based approaches complement each other at a rate of 11\% to 34\%.

\section{Thesis Contributions}
The major contributions of this thesis are as follows:
\begin{itemize}
	\item A comprehensive review of the state of the art in the technical debt field, noting empirical gaps in the theory that enable interested researchers to focus on the main challenges in the field of technical debt.
	\item Enhance knowledge of the technical debt phenomenon by presenting a large-scale empirical study on comment- and metric-based approaches.
	\item Prevent uninformed assumption of technical debt by exposing the risks associated with the practice.

\end{itemize}